\documentclass[12pt,a4paper]{book}

\usepackage[top = 2cm, bottom = 2cm, left = 2.5cm, right = 2.5cm]{geometry}
\usepackage[T1]{fontenc}
\usepackage[brazilian]{babel}
\usepackage[utf8x]{inputenc}
\usepackage{setspace}
\usepackage{color}
\usepackage{multicol}
\usepackage{graphicx}
\usepackage{subfigure}
\usepackage{ftnxtra}
\usepackage{fnpos}
\usepackage{lipsum}

\definecolor{laranja} {RGB} {255, 165, 0}

\onehalfspacing

\begin{document}

\title{Minicurso LaTex}
\author{Murilo Vaz Caetité\footnote{Graduando IO}}
\maketitle

\tableofcontents
\chapter{Introdução}
\label{Intro}
\begin {multicols}{2}
\textcolor {laranja} {Como já se sabe, o ambiente antártico oferece condições extremas de vida, como baixas temperaturas e umidades relativas ainda menores. Para sobreviver nesse ambiente os organismos precisaram desenvolver adaptações bem específicas e isso não é diferente ao tratarmos das aves voadoras que lá vivem.}

Os principais grupos que compõem essas aves são albatrozes, fulmares e “shearwaters”. Essas aves têm locais de nidificação limitados e são em geral confinadas as ilhas subantárticas e a regiões livres de gelo durante o verão austral no continente e na península. Muitas delas são suscetíveis a mudanças ambientais, já que possuem baixas taxas de reprodução e consequente baixo potencial de recuperação populacional, a pesca com espinel, ou pesca de palangre, por exemplo, é uma ameaça para as populações de petréis e albatrozes.

\hrulefill

A maior parte das aves marinhas passa a vida vagando pelo oceano solitariamente à procura de alimento, sendo que grandes grupos podem ser vistos em locais com presas abundantes. Se alimentam geralmente de peixes, lulas, krill e carniça, sendo que cada espécie tem seu modo de obter alimento. Possuem padrões de migração que se estabelecem entre épocas de reprodução, quando nidificam, e outros momentos em que vão em busca de comida ao longo do oceano austral, atlântico, pacífico e índico. Além disso muitas espécies desses grupos são encontradas em ambientes subtropicais próximos, assim como no ártico.
\end {multicols}

\section{Grupos}
\textbf{Albatrozes:}

\begin{itemize}

\item Maior espécie é o albatroz-errante, chegando a 3m de envergadura e pesando de 3kg a 8kg
 %Albatrozes são muito legais
\item Constroem um grande ninho em forma de tigela e colocam um único ovo, sendo que se reproduzem geralmente a cada 2 anos. Das 4 espécies na antártica (que se reproduzem na Geórgia do Sul mais especificamente), apenas uma se reproduz anualmente. Costumam levar 10 anos para atingir maturidade sexual e vivem por muito tempo, podendo chegar até mais de 60 anos. As 4 espécies citadas representam todos os gêneros do grupo no hemisfério sul.  $x=2$, $x*2$, $x/2$, $\frac{x}{x+3}$, $a^{5x}$, $\sqrt[2]{x}$, $\log4$, $\log_{4}3$, $\sin 60$,$\cos 60$, $\tan 60$, $\int 5x$
$\rightarrow$ $\overrightarrow{\Delta r}$ $\left\lbrace\frac{5}{6}\right\rbrace$
\begin{equation}
\lim_{x\rightarrow 0} (x^{3}-2)
\end{equation}
\begin{equation}
\int_{0}^{\infty}f(x)dx=F(b)-F(a)
\end{equation}
\begin{equation}
\vec{F}\:=\:-G\:\frac{m_1\:m_2}{r^2}\:\hat{r}
\end{equation}
\begin{equation}
\left(\begin{array}{lr}
a&b\\
c&d\\
\end{array}
\right)
\end{equation}

\begin{equation}
x=2
\end{equation}
\begin{equation}
\left\lbrace
\begin{array}{cc}
3x + 2y = 6 \\
2x + 3y = 5
\end{array}
\right.
\end{equation}
\begin{eqnarray}
3x + 2y = 6 \\
2x + 3y = 5
\end{eqnarray}
\begin{eqnarray}
\overbrace{ABC} \\
\underbrace{AB} \\
\overline{ab} \\
\underline{cd}\\
\overrightarrow{ABCD}\\
\overleftarrow{ABCD}\\
\widehat{A \cdot B}
\end{eqnarray}

\end{itemize}
\begin{equation}
A+B\:\stackrel{2\:min}{\longrightarrow}\:C+D 
\end{equation}
\begin{tabular}{| l | c | r | c |} \hline
Célula 1 & Célula 2 & Célula 3 & Célula 11 \\ \hline
Célula 4 & Célula 5 & Célula 6 & Célula 11 \\ \hline
Célula 7 & Célula 8 & Célula 9 & Célula 11 \\ \hline
\end{tabular}\\ 

\begin{tabular}{| c | c | c | c |} \hline
\multicolumn{4}{ | c |}{Meses do ano} \\ \hline
Janeiro & Fevereiro & Março & Abril \\ \hline
Maio & Junho & Julho & Agosto \\ \hline
Setembro & Outubro & Novembro & Dezembro \\ \hline
\end{tabular}

\newpage
\begin{figure} [h]
\centering
\includegraphics[scale=0.5]{C:/Users/vcmur/Pictures/Saved Pictures/Kyoka Jiro.png}
\label{ImagemLampada}
\caption{Imagem de uma lâmpada}
\end{figure}

\begin{table}[h]
\centering
\caption{Jogadores mais valiosos do brasileirão 2022}
\vspace{0.5cm}
\begin{tabular}{c|cc}
Nome & Clube & Valor ($km^2$) \\
\hline
1 & Rússia & 17.098.246 \\
2 & Canadá & 9.984.670 \\
3 & China & 9.596.961 \\
4 & Estados Unidos & 9.371.174 \\
5 & Brasil & 8.515.767
\end{tabular}
\end{table}

\begin{flushright}
\begin{minipage}[t]{12cm}
\hrulefill

Trabalho Final de graduação do curso de Análise e desenvolvimento de Sistemas da Faculdade de Tecnologia do Estado de São Paulo
apresentado como requisito para a obtenção do grau de tecnólogo em Análise e desenvolvimento de sistemas.

\hrulefill

\vspace{0.2cm}

{\bf Orientador: Prof. Dr. Fulano}
\end{minipage}
\end{flushright}

\newpage

\begin{titlepage}

\addtolength{\topmargin}{1.5cm}

\textheight = 20cm
\textwidth = 14cm

\setlength{\baselineskip}{1.4\baselineskip}
\begin{center}

{\large NOME DA UNIVERSIDADE}

{\large INSTITUIÇÃO ACADÊMICA OU ESCOLA OU FACULDADE}
\end{center}

\vspace{2cm}
\begin{center}
{\Large\textbf{Título de seu trabalho } }
\end{center}

\vspace{1.5cm}

\begin{center}
{\Large Autor do trabalho}
\end{center}

\vspace{2cm}
\begin{flushright}
\begin{minipage}{10cm}
\hrulefill

Trabalho final de graduação do curso de Análise e desenvolvimento de sistemas da Faculdade de tecnologia
do Estado de São Paulo apresentado como pré requisito para a obtenção do grau de tecnólogo em Análise e
desenvolvimento de sistemas.

\hrulefill

{\textbf{Orientador: Prof.Dr: Fulano}}
\end{minipage}
\end{flushright}
\setlength{\baselineskip}{0.7\baselineskip}
\vfill

\begin{center}
São Paulo

Setembro de 2022
\end{center}

\end{titlepage}

\newpage
\begin{center}
\rule {10cm}{0,01cm}

Murilo Vaz

\vspace{10cm}

\rule {10cm}{0,01cm}

Murilo Vaz 2
\end{center}

\begin{equation} \label{EqTrigonometria}
\sin^2\alpha + \cos^2\alpha = 1
\end{equation}

\begin{equation} \label{EqTrigonometria}
\sin^2\alpha + \cos^2\alpha = 1
\end{equation}

A equação acima \ref{EqTrigonometria} é a primeira relação fundamental da
trigonometria
\newpage
\begin{figure}
\centering
\subfigure[Yamato\label{Yamato}]{\includegraphics[scale =
0.21]{"C:/Users/vcmur/Pictures/Saved Pictures/Yamato.png"}}
\subfigure[Nezuko\label{Nezuko}]{\includegraphics[scale =
0.29]{"C:/Users/vcmur/Pictures/Saved Pictures/Nezuko.png"}}
\center{\subfigure[Planeta Saturno\label{FigSaturno}]{\includegraphics[scale =
0.2]{"C:/Users/vcmur/Pictures/Saved Pictures/Sanji.png"}}}
\caption{Sanji no paraíso\cite{BARBOZA}\label{FigSistemaSolar}
\end{figure}
Nas imagens acima, \ref{Yamato} representa o planeta Terra, \ref{Nezuko} o
planeta vermelho Marte e \ref{FigSaturno} o planeta Saturno, que compõe os anéis.
A imagem acima \ref{FigSistemaSolar}, portanto, contém 3 planetas do Sistema
Solar.

Aprenda, persista e estude a cada dia \cite{BARBOZA}.

\begin{thebibliography}{1}
\bibitem {BARBOZA} BARBOZA, R.C; Curso de LaTeX. Universidade de São
Paulo, 2021.
\end{thebibliography}

\begin{thebibliography}{1}
\bibitem {BARBOZA} BARBOZA, R.C; Curso de LaTeX. Universidade de São
Paulo, 2021.
\end{thebibliography}
\end{document}