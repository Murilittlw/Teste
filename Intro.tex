\chapter{Introdução}
\label{Intro}
\begin {multicols}{2}
\textcolor {laranja} {Como já se sabe, o ambiente antártico oferece condições extremas de vida, como baixas temperaturas e umidades relativas ainda menores. Para sobreviver nesse ambiente os organismos precisaram desenvolver adaptações bem específicas e isso não é diferente ao tratarmos das aves voadoras que lá vivem.}

Os principais grupos que compõem essas aves são albatrozes, fulmares e “shearwaters”. Essas aves têm locais de nidificação limitados e são em geral confinadas as ilhas subantárticas e a regiões livres de gelo durante o verão austral no continente e na península. Muitas delas são suscetíveis a mudanças ambientais, já que possuem baixas taxas de reprodução e consequente baixo potencial de recuperação populacional, a pesca com espinel, ou pesca de palangre, por exemplo, é uma ameaça para as populações de petréis e albatrozes.

\hrulefill

A maior parte das aves marinhas passa a vida vagando pelo oceano solitariamente à procura de alimento, sendo que grandes grupos podem ser vistos em locais com presas abundantes. Se alimentam geralmente de peixes, lulas, krill e carniça, sendo que cada espécie tem seu modo de obter alimento. Possuem padrões de migração que se estabelecem entre épocas de reprodução, quando nidificam, e outros momentos em que vão em busca de comida ao longo do oceano austral, atlântico, pacífico e índico. Além disso muitas espécies desses grupos são encontradas em ambientes subtropicais próximos, assim como no ártico.
\end {multicols}